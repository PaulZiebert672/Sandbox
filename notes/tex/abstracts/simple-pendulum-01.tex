\documentclass{article}
\usepackage{amsmath}
\usepackage{amssymb}
\usepackage{bigints}

\title{Mathematical pendulum}
\author{Paul}
\date{\today}

\begin{document}
\maketitle

\section{Hamiltonian}

\begin{equation}
\mathcal{H} = \dfrac{1}{2} p^2 + 2 \sin^2 \dfrac{1}{2} \theta
\end{equation}

\section{Equations of motion}

Select initial conditions so that $\theta\Big\vert_{t=0} = 0$

\subsection{Librations}

Case $\mathcal{E} < 2$. Assign $k^2 = \dfrac{1}{2} \mathcal{E}$

\begin{eqnarray}
\theta(t) & = & 2 \arcsin (k \, \mathrm{sn}(t ; k)) \\
p(t) & = & 2 k \, \mathrm{cn}(t ; k)
\end{eqnarray}

Period of librations

\begin{equation}
T = 4 \mathrm{K}(k)
\end{equation}

\subsection{Rotations}

Case $\mathcal{E} > 2$. Assign $k^2 = 2 \, \mathcal{E}^{-1}$

\begin{eqnarray}
\theta(t) & = & 2 \, \mathrm{am} (k^{-1} t ; k) \\
p(t) & = & 2 k^{-1} \, \mathrm{dn}(k^{-1} t ; k)
\end{eqnarray}

Period of rotations

\begin{equation}
T = 2 k \, \mathrm{K}(k)
\end{equation}

\section{Elliptic integral via AGM}

The \textit{elliptic integral of the first kind} $\mathrm{F}(\phi;k)$ and the \textit{complete elliptic integral of the first kind} $\mathrm{K}(k)$ are defined as

\begin{eqnarray}
\mathrm{F}(\phi;k) & = & \int \limits_{0}^{\phi} \dfrac{\mathrm{d} \theta}{\sqrt{1 - k^2 \sin^2 \theta}} \\
\mathrm{K}(k) & = & \mathrm{F}\left( \dfrac{\pi}{2}; k \right)
\end{eqnarray}

The complete elliptic integral of the first kind can be evaluated via the \textit{arithmetic-geometric mean} (AGM) 

\begin{equation}
\mathrm{K}(k) = \dfrac{\pi}{2} \dfrac{1}{\mathcal{M}\left( 1, \sqrt{1 - k^2} \right)}
\end{equation}

\section{Jacobi elliptic functions}

Elliptic functions can be represented via Jacobi amplitude function

\begin{eqnarray}
\mathrm{sn}(u;k) & = & \sin \mathrm{am} (u;k) \\
\mathrm{cn}(u;k) & = & \cos \mathrm{am} (u;k) \\
\mathrm{dn}(u;k) & = & \dfrac{\partial}{\partial u} \mathrm{am} (u;k)
\end{eqnarray}

where the amplitude function is defined via inverse of the elliptic integral

\begin{equation}
\mathrm{F}(\mathrm{am}(u; k); k) = u
\end{equation}

\end{document}
