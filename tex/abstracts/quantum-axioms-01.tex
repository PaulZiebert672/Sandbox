\documentclass{article}
\usepackage{amsmath}
\usepackage{amssymb}
\newtheorem{axiom}{Axiom}
\newtheorem{definition}{Definition}
\newtheorem{remark}{Remark}
\DeclareMathOperator{\tr}{Tr}

\title{Axioms of Quantum Mechanics}
\author{Paul}
\date{\today}

\begin{document}
\maketitle

\section{States}

\begin{axiom}
With every quantum system there is associated a complex separable Hilbert space $(\mathfrak{H}, +, \cdot, \langle\diamond,\diamond\rangle)$. The states of the system are all positive trace-class linear maps $\varrho:\mathfrak{H}\rightarrow\mathfrak{H}$ for which $\tr\varrho = 1$
\end{axiom}

\begin{remark}
Almost everywhere it is stated:
The normalized elements $\psi\in\mathfrak{H}$ are the states of the quantum system - it is false.
\end{remark}

\begin{definition}
A state is called a pure state (not pure = mixed) if there exists $\psi\in\mathfrak{H}$ such that

\begin{align*}
\varrho:\mathfrak{H}\rightarrow\mathfrak{H} & , & \alpha \mapsto \varrho(\alpha) = \dfrac{\langle\psi,\alpha\rangle}{\langle\psi,\psi\rangle}\,\psi
\end{align*}

\end{definition}

\begin{remark}
Thus for pure states it is true that state $\varrho$ is associated with the element of the Hilbert space $\psi$.
\end{remark}

\begin{description}
\item[Complex Hilbert space] $(\mathfrak{H}, +, \cdot, \langle\diamond,\diamond\rangle)$
\end{description}

\begin{itemize}
\item $\mathfrak{H}$ is a set that satisfies the axioms of complex vector space

\begin{itemize}
\item $\phantom{\cdot}+:\mathfrak{H}\times\mathfrak{H}\rightarrow\mathfrak{H}$
\item $\phantom{+}\cdot:\mathbb{C}\times\mathfrak{H}\rightarrow\mathfrak{H}$
\end{itemize}

\item Sesquilinear map $\langle\diamond,\diamond\rangle:\mathfrak{H}\times\mathfrak{H}\rightarrow\mathbb{C}$ satisfying

\begin{itemize}
\item $\langle\phi,\psi\rangle = \overline{\langle\psi,\phi\rangle}$ (complex conjugate)
\item $\langle\phi,\psi_1 + \alpha\psi_2\rangle = \langle\phi,\psi_1\rangle + \alpha\langle\phi,\psi_2\rangle$, $\forall \alpha \in \mathbb{C}$
\item $\langle\psi,\psi\rangle \geq 0$, $\forall \psi \in \mathfrak{H}$ and $\langle\psi,\psi\rangle = 0 \Leftrightarrow \psi = 0$
\end{itemize}

\item $\mathfrak{H}$ is complete
\end{itemize}

If one has a sequence in $\mathfrak{H}$, $\phi:\mathbb{N}\rightarrow\mathfrak{H}$, which satisfies the Cauchy property $\forall\epsilon > 0\,\exists N \in \mathbb{N}$ such that $\forall n,m \geq N$: $\|\phi_n - \phi_m\| < \epsilon$, where $\|\phi\| = \sqrt{\langle\phi,\phi\rangle}$, one may already conclude that the sequence $\phi$ converges in $\mathfrak{H}$ i.e. $\exists\phi\in\mathfrak{H}$ such that $\forall\epsilon > 0\,\exists N \in \mathbb{N}$ such that $\forall n \geq N$: $\|\phi - \phi_n\| < \epsilon$.

\begin{remark}
For linear map $A:\mathfrak{H} \supset \mathcal{D}_A \rightarrow \mathfrak{H}$ we will only look at densely defined linear maps

\begin{align*}
\forall\psi\in\mathfrak{H}, \,\forall\epsilon > 0 \,\exists \chi\in\mathcal{D}_A &: & \|\chi - \psi\| < \epsilon \\
A(\phi + \alpha\psi) = A\phi + \alpha A\psi, & & \forall \alpha \in \mathbb{C}
\end{align*}

\end{remark}

\begin{definition}
Positive linear map is a map $A$ such that $\forall\psi\in\mathcal{D}_A$: $\langle\psi, A\psi\rangle \geq 0$
\end{definition}

\begin{definition}
Trace-class linear map is a map $A:\mathfrak{H}\rightarrow\mathfrak{H}$ (defined on the entire Hilbert space) such that $\forall$ orthonormal basis $\{e_n\}$ of $\mathfrak{H}$ the sum/series $\sum_n \langle e_n, A e_n \rangle < \infty$, $\tr A = \sum_n \langle e_n, A e_n \rangle$
\end{definition}

\begin{remark}
Hilbert space $\mathfrak{H}$ has to be separable.
\end{remark}

\section{Observables}

\begin{axiom}
The observables of a quantum system are the self-adjoint linear maps $A:\mathcal{D}_A\longrightarrow\mathfrak{H}$
\end{axiom}

\begin{definition}
A linear map $A:\mathcal{D}_A\longrightarrow\mathfrak{H}$ densely defined on its domain is called self-adjoint if it coincides with its adjoint map $A^{*}:\mathcal{D}_{A^{*}}\longrightarrow\mathfrak{H}$
\end{definition}

\begin{align*}
\mathcal{D}_{A^{*}} = \mathcal{D}_A & & A^{*}\psi = A\psi
\end{align*}

\begin{definition}
The adjoint $A^{*}:\mathcal{D}_{A^{*}}\longrightarrow\mathfrak{H}$ of a linear map $A:\mathcal{D}_A\longrightarrow\mathfrak{H}$ is defined by

\begin{align*}
\mathcal{D}_{A^{*}} = & \,\Big\{ \psi \in \mathfrak{H} \,\Big\vert\, \forall\alpha\in\mathcal{D}_A \,\exists\eta\in\mathfrak{H}: \langle\psi,A\alpha\rangle = \langle\eta,\alpha\rangle \Big\} \\
A^{*}\psi = & \,\eta
\end{align*}

\end{definition}

$A^{*}$ is well defined if there is unique $\eta$

\section{Measurements}

\begin{axiom}
The probability that a measurement of an observable $A$ on a system that is in the state $\varrho$ yields a result in the Borel set $E \subseteq \mathbb{R}$ is given by

\begin{equation*}
\mu_{\varrho}^{A}(E) = \tr\big( P_A(E) \circ \varrho \big)
\end{equation*}

\end{axiom}

$P_A(E)$ is a bounded operator. The composition of trace-class operator with the bounded operator is again trace-class operator.

\begin{equation*}
P_A:\text{Borel($\mathbb{R}$)} \longrightarrow \mathcal{L}(\mathfrak{H}):\text{Banach space of bounded linear maps on $\mathfrak{H}$}
\end{equation*}

is the unique projection-valued measure that is associated with a self-adjoint map $A$ accordingly to the spectral theorem

\begin{equation*}
A = \int\limits_{-\infty}^{\infty} \lambda \, \mathrm{d}P_A(\lambda)
\end{equation*}

\section{Unitary dynamics}

Time intervals $(t_1, t_2)$ during which no measurement occurs

\begin{axiom}
State $\varrho(t_1)$ and state $\varrho(t_2)$ are related through

\begin{equation*}
\varrho(t_2) = \mathcal{U}(t_2 - t_1)\, \varrho(t_1)\, \mathcal{U}^{-1}(t_2 - t_1)
\end{equation*}

where

\begin{equation*}
\mathcal{U}(t) = \exp{\Big(-\dfrac{i}{\hbar} \mathcal{H} t\Big)}
\end{equation*}

\end{axiom}

$\mathcal{H}$ is the energy observable

\begin{equation*}
f(A) = \int\limits_{-\infty}^{\infty} f(\lambda) \, \mathrm{d}P_A(\lambda)
\end{equation*}

\section{Projective dynamics}

Step in when a measurement is made at time $t_m$

\begin{axiom}
The state $\varrho_{after}$ immediately after the measurement of an observable $A$ is

\begin{equation*}
\varrho_{after} = \dfrac{P_A(E) \circ \varrho_{before} \circ P_A(E)}{\tr\big(P_A(E) \circ \varrho_{before} \circ P_A(E)\big)}
\end{equation*}

where $E$ is the smallest Borel set in which the actual outcome of the measurement happened to lie.
\end{axiom}

\end{document}
