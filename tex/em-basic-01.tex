\documentclass{article}
\usepackage{amsmath}
\usepackage{amssymb}
\usepackage{bigints}

\title{Electromagnetism}
\author{Paul}
\date{\today}

\begin{document}
\maketitle

\section{Field equations}

For a given distribution of charges the fields are determined by

\begin{eqnarray*}
\nabla \times \vec{E} & = & - \dfrac{1}{c} \dfrac{\partial}{\partial t} \vec{B} \\
\nabla \cdot \vec{B} & = & 0 \\
\nabla \times \vec{B} & = & \dfrac{1}{c} \dfrac{\partial}{\partial t} \vec{E} + \dfrac{4 \pi}{c} \vec{j} \\
\nabla \cdot \vec{E} & = & 4 \pi \rho
\end{eqnarray*}

Conservation of charge

\begin{eqnarray*}
\dfrac{\partial}{\partial t} \rho + \nabla \cdot \vec{j} & = & 0
\end{eqnarray*}

The motion of the charges is given by

\begin{eqnarray*}
\vec{k} & = & \rho \vec{E} + \dfrac{1}{c} \vec{j} \times \vec{B}
\end{eqnarray*}

\section{Potentials}

It follows that $\vec{B}$ can be represented as $\vec{B} = \nabla \times \vec{A}$, where $\vec{A}$ is called the \textit{vector potential}. Then

\begin{equation*}
\nabla \times \left( \vec{E} + \dfrac{1}{c} \dfrac{\partial}{\partial t} \vec{A} \right) = 0
\end{equation*}

or

\begin{equation*}
\vec{E} + \dfrac{1}{c} \dfrac{\partial}{\partial t} \vec{A} = - \nabla \phi
\end{equation*}

where $\phi$ represents a scalar function --- the \textit{scalar potential}. Then values of fields can be expressed as

\begin{eqnarray*}
\vec{B} & = & \nabla \times \vec{A} \\
\vec{E} & = & - \nabla \phi - \dfrac{1}{c} \dfrac{\partial}{\partial t} \vec{A}
\end{eqnarray*}

Making use of the general vector relation $\nabla \times \nabla \vec{A} = \nabla \nabla \cdot \vec{A} - \nabla^2 \vec{A}$ differential equations for potentials can be written in the form

\begin{eqnarray*}
\dfrac{1}{c^2} \dfrac{\partial^2}{\partial t^2} \vec{A} - \nabla^2 \vec{A} + \nabla \left( \nabla \cdot \vec{A} + \dfrac{1}{c} \dfrac{\partial}{\partial t} \phi \right) & = & \dfrac{4 \pi}{c} \vec{j} \\
-\nabla^2 \phi - \dfrac{1}{c} \dfrac{\partial}{\partial t} \nabla \cdot \vec{A} & = & 4 \pi \rho
\end{eqnarray*}

\subsection{Gauge transformation}

The vector $\vec{A}$ is not completely defined by the value of $\vec{B}$. Since for any scalar function $\chi$ it is true that $\nabla \times \nabla \chi = 0$, transformation

\begin{eqnarray*}
\vec{A} & \rightarrow & \vec{A} + \nabla \chi \\
\phi & \rightarrow & \phi - \dfrac{1}{c} \dfrac{\partial}{\partial t} \chi
\end{eqnarray*}

does not change the values of fields $\left( \vec{E}, \vec{B} \right)$.

\subsection{Lorentz gauge}

The condition

\begin{equation*}
\nabla \cdot \vec{A} + \dfrac{1}{c} \dfrac{\partial}{\partial t} \phi = 0
\end{equation*}

is called the \textit{Lorent gauge}. Equations for potentials then become

\begin{eqnarray*}
\dfrac{1}{c^2} \dfrac{\partial^2}{\partial t^2} \vec{A} - \nabla^2 \vec{A} & = & \dfrac{4 \pi}{c} \vec{j} \\
\dfrac{1}{c^2} \dfrac{\partial^2}{\partial t^2} \phi - \nabla^2 \phi & = & 4 \pi \rho
\end{eqnarray*}

There is still freedom in selecting $\chi$ which satisfies the homogeneous wave equation

\begin{equation*}
\dfrac{1}{c^2} \dfrac{\partial^2}{\partial t^2} \chi - \nabla^2 \chi = 0
\end{equation*}

\subsection{Coulomb gauge}

Another important gauge which is particularly important in quantum theory is

\begin{equation*}
\nabla \cdot \vec{A} = 0
\end{equation*}

which is called the \textit{Coulomb gauge}. Equations for potentials then become

\begin{eqnarray*}
\dfrac{1}{c^2} \dfrac{\partial^2}{\partial t^2} \vec{A} - \nabla^2 \vec{A} + \dfrac{1}{c} \dfrac{\partial}{\partial t} \nabla \phi & = & \dfrac{4 \pi}{c} \vec{j} \\
- \nabla^2 \phi & = & 4 \pi \rho
\end{eqnarray*}

There is still freedom in selecting $\chi$ which can be any harmonic function

\begin{equation*}
\nabla^2 \chi = 0
\end{equation*}

\section{Retarded potential}

Special solutions for inhomogeneous wave equations are

\begin{eqnarray*}
\vec{A}(t,x) & = & \dfrac{1}{c} \bigintss \dfrac{j\left(t - \dfrac{r_{x x'}}{c}, x'\right)}{r_{x x'}} \, \mathrm{d} \tau' \\
\phi(t,x) & = & \bigintss \dfrac{\rho\left(t - \dfrac{r_{x x'}}{c}, x'\right)}{r_{x x'}} \, \mathrm{d} \tau'
\end{eqnarray*}

The Lorentz gauge is satisfied by the charge conservation condition.

For the part of the field, which satisfies the homogeneous wave equation, on can choose $\chi$ within the Lorentz gauge so that $\phi$ vanishes. The field which is independent of charges is given by

\begin{eqnarray*}
\dfrac{1}{c^2} \dfrac{\partial^2}{\partial t^2} \vec{A} - \nabla^2 \vec{A} & = & 0 \\
\nabla \cdot \vec{A} & = & 0 \\
\vec{E} & = & - \dfrac{1}{c} \dfrac{\partial}{\partial t} \vec{A} \\
\vec{B} & = & \nabla \times \vec{A}
\end{eqnarray*}

\end{document}
